%-------------------------------------------------------------------------------
%	SECTION TITLE
%-------------------------------------------------------------------------------
\cvsection{Summary}


%-------------------------------------------------------------------------------
%	CONTENT
%-------------------------------------------------------------------------------
\begin{cvparagraph}

%---------------------------------------------------------
%https://cdn.uconnectlabs.com/wp-content/uploads/sites/74/2019/08/CV-cancer-bio-sample.pdf

As a sixth-year postdoctoral researcher at the University of Kansas and an active member of the CMS experiment at the LHC, I have established myself as an emerging leader in \textbf{High-Energy Particle and Nuclear Physics} and \textbf{Physics Instrumentation and Detectors}. My work has consistently pushed the CMS physics program beyond its original design envelope, showcasing both the breadth of its physics reach and previously unforeseen capabilities. Several of these contributions have received \textbf{high-profile international recognition}: they have been selected as ``Editors’ Suggestions'' by leading journals~\cite{Sirunyan:2017xku,PRL_edit,CERNCourier,CMS:2022arf,CMS:2023iam}, described by the Head of the MIT Physics Department as belonging to the “never thought I’d see it” category~\cite{MIT}, and highlighted in a \textbf{CERN press release}~\cite{CERNRelease} that was widely disseminated across research institutions~\cite{LIP} and major science news outlets~\cite{physorg}. These accomplishments reflect my \textbf{ability to strategically plan, execute, and lead ambitious research projects} that both advance the field and expand what our experimental infrastructure is thought to be capable of delivering.

My scientific leadership is reflected in \textbf{three major awards}, including recognition as a \textbf{Distinguished Researcher} by both CMS and the University of Kansas~\cite{Award2,Award3}, as well as an award for my pioneering work in \textbf{luminosity calibration}~\cite{Award1}. My PhD thesis~\cite{Krintiras:2018oom} was nominated for a \textbf{best thesis award}, and my alma mater (UCLouvain) recently named me among its \textbf{notable alumni}~\cite{Award4}. Within CMS, my expertise has been repeatedly acknowledged through \textbf{leadership roles} in major physics and calibration groups (HIN, HIG, TOP, EGM, JME, LUM)~\cite{POG,PAG}, and most recently through my appointment to the \textbf{CMS Publication Committee}, which is entrusted with overseeing the quality and impact of the collaboration’s scientific output. In parallel, I have helped to shape the \textbf{community’s scientific agenda} by co-organizing internally funded workshops within CMS and taking active roles in the organization of international conferences and topical workshops. At KU, I have also played an active role in shaping \textbf{internal funding priorities across three proposal cycles}, helping to direct resources toward emerging scientific opportunities.


Over the years, I have represented CMS at major international conferences, served as co-author on numerous publications, and acted as peer-reviewer for leading journals. These experiences, combined with the opportunity to mentor young researchers, have allowed me to build a strong profile at the intersection of physics analysis, detector calibration, and international collaboration management.

\iffalse
Groups that are interested in hiring me can \textbf{greatly benefit} from close collaboration with many teams around the world either
remotely (by definition a fundamental part of our job within international collaborations) or in person (for my case, already
counting five countries and eight institutes in total). Based on my in-depth analysis, also including detector operating
and physics-object calibration activities, and managerial knowledge, we can \textbf{seed and lead efforts} in designing, executing,
documenting, and convening physics analyses and groups \textbf{within large experimental collaborations}. In parallel, synergies in the field of \textbf{data science and heavy ion phenomenology}, and engagement to existing projects for \textbf{future colliders} are foreseen, as I documented in a series of corresponding research papers.
\fi
\end{cvparagraph}